\documentclass[12pt]{article}
\usepackage{amsmath}

\begin{document}


\section{0.5}

For values of $r$ less than 1, the equations have only one fixed point, the origin, which is attracting. This leads to no-chaotic behaviour as all trajectories tend to it.

\section{9}

At $r = 1$ a supercritical pitchfork bifurcation takes place causing the origin to become unstable and two stable attracting points $C1$ and $C2$ to appear, these are the loci of the system for all higher $r$ values. The points remain globally attracting until $r = 13.926...$.

\section{13.926}
At $r=13.926...$ the system experiences transient chaos, as the an unstable limit cycle moves from the origin towards $C1$ and $C2$. Thus trajectories will orbit chaotically around the two points until eventually staring to approach one asymptomatically, the red curve is a chaotically orbiting trajectory and the green is one started much closer to an attracting point and thus is already approaching it.

\section{24.06}
At $r=24.06$ the transient chaos ceases and the system becomes a strange attractor. This is due to the unstable limit cycles becoming so close to the stable points that trajectories can no longer cross it. Even though the green curve starts very close to an attracting point it is still repelled and eventually participates in the strange attractor.

\section{24.74}
At $r=24.74$ the unstable limit cycle reaches the stable points resulting in a supcritical hopf bifurcation and all points become unstable. $r=28$ is the value studied most intently by Lorenz and is known as the Lorenz attractor; 28 was picked since it is a round number in the low end of the regime.

\section{100}
As $r$ continues to increase the number of periodic orbits grows to infinity at around 100 then decreases for higher values this type of behaviour characterizes much of what is known about large $r$ values although much is still unknown. For comparison the Lorenz attractor ($r=28$ is plotted in green, the similarity between the two is striking.

\end{document}
